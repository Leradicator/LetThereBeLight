%%%%%%%%%%%%%%%%%%%%%%%%%%%%%%%%%%%%%%%%%%%%%%%%%%%%%%%
% % Lines starting with % are comments, which are ignored.
% % This is a handy way of indicating the date and version of
% % your document, to wit:
% %
% % LaTeX sample file
% % Modified March, 2002
% %
%%%%%%%%%%%%%%%%%%%%%%%%%%%%%%%%%%%%%%%%%%%%%%%%%%%%%%%
% % Title and author(s)
%%%%%%%%%%%%%%%%%%%%%%%%%%%%%%%%%%%%%%%%%%%%%%%%%%%%%%%
\title{A case study for expanding the complex plane
        to a more attractive mathematical set}
\author{Issam Outassourt\thanks{
                    Bashar
                  }\thanks{
                    Marut    
                  }\thanks{
                    and GOD above all.
                  }
        }
%%%%%%%%%%%%%%%%%%%%%%%%%%%%%%%%%%%%%%%%%%%%%%%%%%%%%%%
\documentclass{article}
%%%%%%%%%%%%%%%%%%%%%%%%%%%%%%%%%%%%%%%%%%%%%%%%%%%%%%%
% %
% % The next command allows your in import encapsulated
% % postscript files, .epsf or .eps files, which
% % contain vector graphic image data.
% %
%%%%%%%%%%%%%%%%%%%%%%%%%%%%%%%%%%%%%%%%%%%%%%%%%%%%%%%
\usepackage{graphicx}
\usepackage{cjhebrew}
\usepackage{amsfonts}
%%%%%%%%%%%%%%%%%%%%%%%%%%%%%%%%%%%%%%%%%%%%%%%%%%%%%%%
% % We use newtheorem to define theorem-like structures
% %
% % Here are some common ones. . .
%%%%%%%%%%%%%%%%%%%%%%%%%%%%%%%%%%%%%%%%%%%%%%%%%%%%%%%
\newtheorem{theorem}{Theorem}
\newtheorem{lemma}{Lemma}
\newtheorem{proposition}{Proposition}
\newtheorem{scolium}{Scolium}   %% And a not so common one.
\newtheorem{definition}{Definition}
\newenvironment{proof}{{\sc Proof:}}{~\hfill QED}
\newenvironment{AMS}{}{}
\newenvironment{keywords}{}{}
%%%%%%%%%%%%%%%%%%%%%%%%%%%%%%%%%%%%%%%%%%%%%%%%%%%%%%%
% %   The first thanks indicates your affiliation
% %
% %  Just the name here.
% %
% % Your mailing address goes at the end.
% %
% % \thanks is also how you indicate grant support
% %
%%%%%%%%%%%%%%%%%%%%%%%%%%%%%%%%%%%%%%%%%%%%%%%%%%%%%%%


\begin{document}
\newpage
\maketitle
%%%%%%%%%%%%%%%%%%%%%%%%%%%
% abstract, keywords and Subject classification are optional.
%%%%%%%%%%%%%%%%%%%%%%%%%%%
\begin{abstract}
    The aim of this document is to challenge the common fact within the
    mathematical world claiming that $\mathbb{C}$ is as complete as we
    would think about it. This paper does not deny any commonly known
    properties of the set such that ``$\mathbb{C}$ is an algebraic closure
    of $\mathbb{R}$'' but rather blame the mathematic community for not
    trying to extend this set and account for solving problems like the
    following equation,
    $$ z^{*}z = -1 $$
    and their interest in widening the range of possibilities in modern
    daily life mathematics.
\end{abstract}

% Most people don't use these, so they are "commented out"
% by starting the lines with a "%"
%\begin{keywords}
%   \LaTeX, typesetting
%\end{keywords}

%\begin{AMS}
%   50C60, 18C25
%\end{AMS}

%%%%%%%%%%%%%%%%%%%%%%
% % Here is the start of the Text
%%%%%%%%%%%%%%%%%%%%%%
\section{Introduction}
While maintaining the consistency of the complex plane, we wish to introduce
an innovative approach to construct a set that solves for equations of the type,

\begin{equation}
    z^{*}z = -1 
\end{equation}

where $^{*}$ denotes the conjugation sign known in $\mathbb{C}\ as\ (a+ib)^{*} = (a-ib)$.
We will first briefly discuss the assumptions that denote the set that contains
solutions to the equation $(1)$, assumptions which are regarded as axiomatic definitions.

\begin{definition}
    Let there be $\mathbb{L}$ such that $(1)$ admits at least one solution,
    $l \in \mathbb{L}$

    Following are a list of assumptions that hopefully characterize the set $\mathbb{L}$
    \begin{enumerate}
        \item $\mathbb{C}\ \subset\ \mathbb{L}$
        \item $(\mathbb{L},+,.)$ is a vector space over the field $\mathbb{C}$
        \item $(\mathbb{L},+,\cdot)$ is a field
        \item The conjugation operator $^{*}$ keeps the same known qualities:
            \begin{itemize}
                \item distributivity over $+$: $(a+b)^{*}=a^{*}+b^{*}$
                \item distributivity over $\cdot$: $(a \cdot b)^{*} = a^{*}\cdot b^{*}$
                \item ivolutivity: $a^{**} = a$
            \end{itemize}
    \end{enumerate}
\end{definition}


\begin{proposition}
    From the above $1.3$ it follows that equation $(1)$ does not admit
    a unique solution. There are in fact infinitely many:
    $$\forall \phi \in \mathbb{R}, (e^{i\phi}\cdot l)^{*}\cdot (e^{i\phi}\cdot l) = -1 $$
    $$\forall n \in \mathbb{Z}, (l^{2n+1})^{*}\cdot (l^{2n+1}) = -1 $$

    A justification for the second clause follows from the following result of $1.1$
    that is, ``$l$ is not a zero of any element of $\mathbb{C}[X]$''. Otherwise,
    $l$ would be in $\mathbb{C}$\\
    For similar reasons, the following property holds as well:
    $$ l^{*} \neq l $$ 
\end{proposition}

\section{Announcing the traits of the set}
Following our introduction, we want to discuss some traits of the set
$\mathbb{L}$ in the following section and show some properties that do not
need formal proofs in that they are easy to understand.\\
First, let us witness from $Proposition\ 1$ and earlier results that $l$
is not distinguishable from any of its odd powers, nor of its phases with
respect to equation $(1)$. We will further see to it that we solve $(1)$ in
$\mathbb{L}$ and derive the solutions for equations of the type $y^{*}y = x$
with generic values of $x$, yet being open to complete $\mathbb{L}$ in that respect if needed.
Regarding the dimensionality of $\mathbb{L}$ being a field as well as a $\mathbb{C}$ vector space,
we can derive from the above results that $(l^{n})_{n\in \mathbb{Z}}$ is a linearly independent family.
Thus $\mathbb{L}$ is infinite-dimensional. $\mathbb{C}$ being algebraically closed, together with the
assumed field structure of $\mathbb{L}$ guarantees that property.

From $l^{*}=-l^{-1}$, we can consider that $(l^{n})_{n\in \mathbb{Z}} \sim (\cdots,{l^{*}}^2,l^{*},1,l,l^2,\cdots)$.
This family does not necessarily constitute a base for the vector space $\mathbb{L}$ as we will guide ourselves through
solving the following set of equations
\begin{equation}
    y^{*}y = x\ |\ x \in \mathbb{C}
\end{equation}
considering the solutions to be elements of $\mathbb{L}$ and we will see 
what implications there are regarding the structure of this set.

\begin{cjhebrew}
    way*o'mEr 'E:lohiym\, y:hiy 'wor way:hiy--'wor;
\end{cjhebrew}

\section*{About the author:}
Suffices to know that my testament holds enough information about me.
All thanks are due to God, my Father and your Father.
My discoveries concerning this set go far beyond what has been laid down in this document and both Marut
and Bashar were of great help during this process. I thank them both for what they have done.





\end{document}
